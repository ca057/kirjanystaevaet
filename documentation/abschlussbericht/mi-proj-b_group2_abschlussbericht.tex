\documentclass{lni}
\let\ifpdf\relax

\IfFileExists{latin1.sty}{\usepackage{latin1}}{\usepackage{isolatin1}}

\usepackage{listings}
\usepackage{graphicx}
\usepackage{hyperref}

\lstset{basicstyle=\ttfamily,breaklines=true}

\author{
	Christian Ost, Madeleine Rosenhagen, Ludwig Thormann, Johannes Trepesch \\ 
	\\ 
	Abteilung \\ 
	Einrichtung \\ 
	Anschrift \\ 
	Postleitzahl Ort \\ 
	emaiaddresse@autor1 \\
	emaiaddresse@autor2
}
\title{Abschlusdokumentation -- kirjanystaevaet}


\begin{document}
\maketitle
\tableofcontents
\newpage

\begin{abstract}
Es folgt ein kurzer �berblick �ber die Arbeit.
\end{abstract}

\section{Einleitung}

\section{Systemarchitektur}

\section{Umsetzung des Anforderungskatalogs}
	\subsection{Datenmanagement}
	Dies ist ein Dummytext um Commit auszuprobieren. Noch ein Test um GitGui auszuprobieren
	\subsection{REST-API}
	Die REST-API liefert Bestandsdaten im JSON-Format �ber die zum Verkauf stehenden B�cher. Tabelle \ref{rest-api} gibt Auskunft �ber das verwendete URL-Schema, dabei muss der angegebenen URL \lstinline|http://localhost:8080/kirjanystaevaet/api/v1| vorangestellt werden.
	
	\begin{table}[h]
		\caption{URL-Schema der REST-API}
		\begin{tabular}{|c|c|c|c|}
			\hline
			URL & (optionale) Parameter & Methode & Beschreibung \\ \hline \hline
			\lstinline|/books| & \lstinline|limit (long)| & \lstinline|GET| & alle verf�gbaren B�cher \\ \hline
			\lstinline|/books/{category}| & \lstinline|limit (long)| & \lstinline|GET| & alle verf�gbaren B�cher der �bergebenen Kategorie \\ \hline
			\lstinline|/books/{isbn}| & & \lstinline|GET| & das Buch mit der �bergebenen ISBN \\ \hline
		\end{tabular} 
		\label{rest-api}
	\end{table}
	
\section{User Stories}
	\subsection{Registrierung und Login}
	\dots
	
	Da es zu einer Postleitzahl mehrere Orte geben kann, zum Beispiel bei \lstinline|37627|, wird dem Nutzer nach Eingabe einer Postleitzahl eine Liste mit allen zutreffenden Ortsnamen angezeigt. Aus dieser ist der korrekte Ort auszuw�hlen. Die m�glichen Orte werden im Hintergrund �ber eine AJAX-Abfrage vom Server geholt und als Radio-Buttons in die HTML-Seite eingef�gt.
	
	\dots
	
	\subsection{Buchauswahl und Bestellung}
	
	\subsection{Administration des Web-Shops}

\section{Zusammenarbeit}
	\begin{itemize}
		\item git
		\item GitHub-Issues
		\item Austausch/Updates per Mail
		\item w�chentliche Treffen
	\end{itemize}
	
\section{Lessons Learned}
	\begin{itemize}
		\item[Git] Wie l�sen wir Merge-Konflikte? Wie teilen wir die Arbeit auf unterschiedliche Branches aus? Wie halten wir unseren Master einigerma�en fehlerfrei?
		\item[Naming Conventions] Wie benennen wir unsere Feature-Branches? Wie benennen wir unsere Klassen? Wie benennen wir unsere Packages?
	\end{itemize}
	
\bibliography{lnitemplate}

\end{document}
