\section{Zusammenarbeit}

Bei der Anfertigung von kirjanystaevaet mussten viele Anwendungsfälle und Problemstellungen gemeinsam gelöst werden. Mit folgenden Ansätzen haben wir versucht, unsere Zusammenarbeit effektiv zu gestalten:

\paragraph{Wöchentliche Treffen} Fast durch das gesamte Semester standen wöchentliche Treffen an, häufig auch mit den Betreuern. Vor allem gegen Ende der Bearbeitungszeit wurden diese Termine ausgeweitet und viel Code gemeinsam produziert. Dadurch konnten viele Fragen geklärt sowie Anforderungen von anderen Stellen des Codes eingearbeitet werden.

\paragraph{Feature-Branches} Da eine in ihren Grundzügen funktionsfähige Version elementar für die meisten Implementierungen war, wurden neu zu implementierende Features auf andere Branches ausgelegt und erst nach (mehr oder weniger) erfolgreichem Testen auf dem Master-Branch zur Verfügung gestellt. Dadurch konnten wir Zeitverluste aufgrund von Bugs anderer Teammitglieder gering halten. 

	\begin{itemize}
		\item GitHub-Issues
		\item Austausch/Updates per Mail
	\end{itemize}