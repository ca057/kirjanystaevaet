\section{User Stories}
	\subsection{Registrierung und Login}
	\dots
	
	Da es zu einer Postleitzahl mehrere Orte geben kann, zum Beispiel bei \lstinline|37627|, wird dem Nutzer nach Eingabe einer Postleitzahl eine Liste mit allen zutreffenden Ortsnamen angezeigt. Aus dieser ist der korrekte Ort auszuwählen. Die möglichen Orte werden im Hintergrund über eine AJAX-Abfrage vom Server geholt und als Radio-Buttons in die HTML-Seite eingefügt.
	
	\dots
	
	\subsection{Buchauswahl und Bestellung}
	\subsection{Kund\_innen möchten die Bücher ihrer Wahl bestellen}
	 Ist man als User\_in eingeloggt und hat sich für ein Buch entschieden, steht als nächstes die Bestellung des Buches an. Zwischen der Auswahl und dem eigentlichen Bestellvorgang ist die Ablage des gewünschten Buches in einen Warenkorb vorgesehen, in dem beliebig viele Bücher gesammelt und bestellt werden können. Zudem bietet der Warenkorb eine Auflistung der "Books of Interest" mit dazugehörigen Informationen, sodass sie Kund\_innen vor der Bestellung einen Überblick darüber verschaffen können, welche Bücher bestellt werden sollen. Folgende User-Stories sollten also von einem Warenkorb bearbeitet werden können:
	 <Als Kund\_in möchte ich ein Buch in den Warenkorb legen, damit ich es bestellen kann.>
	 <Als Kund\_in möchte ich einen Überblick über die Bücher haben, die ich bestellen möchte.>
	 und folglich <Als Kund\_in möchte ich eine Übersicht über den Gesamtpreis meiner Bestellung erhalten.>
	 <Als Kund\_in möchte ich eine Bestellung aufgeben>
	 
	\subsection{Administration des Web-Shops}
