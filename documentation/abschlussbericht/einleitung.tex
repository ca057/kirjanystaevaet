\section{Einleitung}
	Aufgabenstellung des hier dargestellten Projekts ist die Umsetzung eines Webshops unter Nutzung von Spring\footnote{\hyperlink{http://spring.io/}{http://spring.io/}}. Spring per se ist ein Framework zur Unterstützung bei der Programmierung von Java-basierten Anwendungen. Durch Spring MVC ist dieses Kerngebiet auch auf Webanwendungen erweiterbar. Ziel ist es, Progammierer\_innen neben der allgemeinen Erstellung von funktionierendem Code auch die Einhaltung und Implementierung von Konzepten wie dependenciy injections und separations of concerns zu erleichtern. 
	
	Anforderung an das zu implementierende Programm war ein Web-Shop, der datenbankbasiert Bücher und Nutzer\_innenkonten verwalten sowie Bestellungen entgegennehmen kann. Zur Erleichterung der Bedienung sollten Nutzer\_innen eine Suchfunktion sowie eine Bestellübersicht, Administrator\_innnen ein Verwaltungsbackend samt Statistiken zur Verfügung stehen. Zur Interaktion mit anderen Shops war eine REST-API gewünscht.
	
	Im Zeitraum von Oktober 2015 bis März 2016 enstand aus diesen Anforderungen \textit{kirjanystaevaet} (fin. "`Buchfreund\_in"'), ein thematisch ungebundener -- aufgrund der Buchvorgaben jedoch zu Beginn informatisch geprägter -- Buchshop. Die im Verlaufe der Implementierung getroffenen Designentscheidungen sowie aufgetretene Probleme und Fragestellungen werden in diesem Bericht ausgeführt. Auf eine allgemeine Einführung in die Systemarchitektur folgt eine detailierte Beschreibung zur Umsetzung der einzelnen Punkte des Anforderungskataloges, von Modellierung der Datenbank bis hin zur Nutzungsoberfläche. Abschluss bildet eine kritische Einschätzung des fertigen Produkts. 