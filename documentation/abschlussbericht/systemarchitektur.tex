\section{Systemarchitektur}

\subsection{Packagestruktur}

Bei der Aufteilung der Klassen stand die Idee der \textit{seperation of concerns} im Vordergrund. In der Implementierung spiegelt sich also der Versuch wider, durch die Struktur möglichst deutliche, voneinander unabhängige und damit leichter austauschbare Schichten zu erzeugen.


\begin{figure}[h]
	\centering
	\includegraphics[width=\linewidth]{files/packages}
	\caption{Die Packagestruktur von \textit{kirjanystävät}}
	\label{fig:packages}
\end{figure}

Das System ist dabei in Anwendung (\lstinline|appl|) und Web (\lstinline|web|) unterteilt (siehe dazu auch Abbildung \ref{fig:packages}). Damit soll die klare Trennung von Programmlogik und Userinterface hervorgehoben werden. Datenbankbezogene Packages befinden sich ebenfalls im Anwendungsteil, könnten aber auch als eigenes Package ausgelagert werden. Aufgrund der engen Verbindung von Anwendung zur Datenbank (und umgekehrt) ist dies jedoch nebensächlich. Extra stehen die Konfigurationsklassen sowie eigene Exceptions.

Im Anwendungspackage herrscht eine vorwiegende Zweiteilung zwischen datenbankbezogenen (data) und datenverarbeitenden (logic) Klassen. Zusätzlich existiert ein Adminpackage, das für erste Initialisierungen des Programms zuständig ist. Data ist -- wie bei Hibernateanwendungen üblich -- in ein Package mit Entities (items) und den transaktionalen Zugriffsklassen (DAO) unterteilt. Hilfsklassen zum Aufbau von Entityobjekten sind im Builderpackage zu finden. Logic wiederrum enthält die ebenfalls konventionellen Serviceklassen, die den Zugriff auf DAOs verwalten, sowie ein Securitypackage, das die Implementation von Spring-Klassen für Login etc. enthält. Durch diese Aufteilung des Appl-Packages soll der Zugriff auf DAO-Klassen reglementiert und außerhalb der Services unnötig gemacht werden.

Das Package \lstinline|web.controllers| stellt die Schnittstelle zwischen den Nutzer\_innen und der Anwendungslogik dar. Auf oberster Ebene liegen neben allen Interfaces Beans für die Fehlerbehandlung sowie für das Setzen von globalen Attributen. Das Package ist wiede\-rum unterteilt in die Bereiche
\begin{itemize}
	\item[Frontend] Darstellung aller Views, für die keine Authentifizierung notwendig ist, sowie aller Views für eingeloggte Nutzer\_innen mit der Rolle \lstinline|USER|
	\item[Backend] Darstellung aller Views für die Verwaltung des Shops, Authentifizierung als \lstinline|ADMIN| erforderlich
	\item[API] Verwaltung der als API deklarierten URLs, Zugriff ohne Authentifizierung möglich
\end{itemize}
Die Controller sind nach ihrem primären Einsatz einem Bereich zugeordnet, der Zugriff und die Authentifizierung wird in der Security-Konfigurationsklasse geregelt. Darüber hi\-naus werden in einem weiteren Sub-Package Hilfsklassen beziehungsweise -Beans gehalten.

\subsection{Allgemeine Grundsätze}

Neben dem oben bereits angesprochenem \textit{separation of concerns} ist das Programm komplett auf Java-Konfiguration gestützt und verzichtet auf die Einbindung von XML-basierten Dateien. Dies hat den Vorteil, dass Java im Gegensatz zum XML type-safe ist, wodurch viele Fehler bereits durch den Compiler entdeckt werden können, und -- vor allem in Verbindung mit IDEs -- eine flexiblere Navigation zwischen einzelnen Dateien bietet. Nachteil dagegen sind die im Internet zu findenden Lösungsvorschläge, die zum Großteil auf XML basieren.

Eng damit verbunden ist die Verwendung von Annotationen, wo immer möglich. Dadurch ist in den jeweiligen Javaklassen sofort ersichtlich, welchen Bestandteil die jeweilige Klasse im Programm erfüllen soll. Zudem sind unübersichtliche Konfigurationsdateien obsolet, da ein Componentscan die angegebenen Packages nach als Bean definierten Klassen durchsucht. Ebenfalls auf Annotationen basiert die Injection von Beans, die auf zwei Zeilen -- die Angabe der Variable samt darüber stehendem \texttt{@Autowired} -- beschränkt ist.

Ebenfalls an mehreren Stellen zum Einsatz kommen sollten Neuerungen aus Java 8. Dazu zählen zum einen Lambda Expressions, mit denen Funktionen als Parameter -- zum Beispiel von Streams -- übergeben werden, zum anderen \texttt{Optionals}, mit denen das Vorkommen von \texttt{null} auf ein Minimum beschränkt wird.